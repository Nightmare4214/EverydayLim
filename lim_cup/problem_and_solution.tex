\documentclass[a4paper]{article}
\usepackage{ctex}% 中文支持
\usepackage{geometry}% 用于页面设置
\usepackage[dvipsnames, svgnames, x11names]{xcolor} % 颜色支持
\usepackage[
colorlinks=true,
linkcolor=Navy,
urlcolor=Navy,
citecolor=Navy,
anchorcolor=Navy
]{hyperref}
\usepackage{tcolorbox}% 支持更好的文本框
\tcbuselibrary{skins, breakable}% 支持文本框跨页
\usepackage[english]{babel}% 载入美式英语断字模板
\usepackage{subfigure}
\usepackage{amsmath}
\usepackage{fancyhdr}
\usepackage{tasks}
\usepackage{esint}
\usepackage{pifont}
\usepackage{lmodern}
\pagestyle{fancy}
\lhead{}
\chead{lim杯}
\rhead{}
\lfoot{作者Nightmare4214}
\cfoot{\thepage}
\rfoot{}
\renewcommand{\headrulewidth}{0.4pt}
\renewcommand{\footrulewidth}{0.4pt}

\newtcolorbox{mybox}[1]{enhanced, colback=GhostWhite, colframe=LightGray, coltitle=black,fonttitle=\bfseries\Large, bottomrule=1ex, breakable=true,title=#1}

\begin{document}
{\centering\section*{lim杯数学方面综合比赛(试题)}}
\noindent
\textbf{注意事项:}\\
\begin{enumerate}
    \item \text{建议闭卷,如果自己开卷莉姆丝在看着你。}
    \item \text{内容以高数、线代和概率论为主,另有文学类题目。}
    \item \text{答题后将答案(仅答案,无需试题原题,注意标明题号)发送至邮箱}\\
    \textcolor{blue}{2280883416@qq.com}\text{(即莉姆丝账号的QQ邮箱),暂不接受其他方式提交。如果有莉}\\
    \text{姆丝粉丝牌记得截图(等级)发过来,有小加分(1级1分,上限10分),附在提交的答}\\
    \text{案尾页}
    \item \text{本卷正式题目总分250分,附加题(小作文)50分,共计300分。如果只想做附加题,}\\
    \text{在发送的答案前标明,将按照满分300分对附加题进行评分。}
    \item \text{友谊第一,比赛第二。}
\end{enumerate}
\textbf{参考公式:}\\
醒醒,公式都不记还想我给你写?\\
\textbf{零、成分查询题(共一题,答对不计分,答错扣300分}\\
在这里该单推的粉毛是谁?\\

\begin{figure}[ht]
    \begin{minipage}{0.24\linewidth}
        \centerline{\includegraphics[width=1\textwidth]{taffy.PNG}}
        \centerline{A.永雏塔菲}
    \end{minipage}
    \begin{minipage}{0.24\linewidth}
        \centerline{\includegraphics[width=1\textwidth]{aqua.PNG}}
        \centerline{B.阿夸}
    \end{minipage}
    \begin{minipage}{0.24\linewidth}
        \centerline{\includegraphics[width=1\textwidth]{hirro.PNG}}
        \centerline{C.hirro}
    \end{minipage}
    \begin{minipage}{0.24\linewidth}
        \centerline{\includegraphics[width=1\textwidth]{lim.PNG}}
        \centerline{D.莉姆丝}
    \end{minipage}
    % \caption*{都是表情图}
\end{figure}

\newpage
{\centering\section*{lim杯数学方面综合比赛(试题)}}
\noindent
\textbf{一、选择题(每题6分共10题,共计60分)}\\
在每小题给出的四个选项中,只有一项是符合题目要求的。\\
1.设m与n都是常数,若反常积分$\int_{0}^{+\infty} \frac{x^n(1-e^{-x})}{(1+x)^{m}}\mathrm{d}x$收敛,则m与n的取值\\
范围为(\quad).\\
A.$n>-2,m>n+1$\quad B.$n>-2,m<n+1$\quad C.$n<-2,m<n+1$\quad D.$n<-2,m>n+1$\\
\textbf{解:}
\begin{align*}
	\lim \limits_{x \to 0^{+}} x^{-(n+1)}\: \frac{x^n(1-e^{-x})}{(1+x)^m}=1
\end{align*}
$$\Rightarrow -(n+1)<1$$
$$\Rightarrow n>-2$$
\begin{align*}
	\lim \limits_{x \to +\infty} x^{m-n}\:\frac{x^n(1-e^{-x})}{(1+x)^m}=1
\end{align*}
$$\Rightarrow m-n>1$$
$$\Rightarrow m>n+1$$
$$\Rightarrow \text{选A}$$
2.在曲线$x=t,y=-t^2,z=t^3$的所有切线中,与平面$27x+54y+27z=4$平行的切线(\quad).\\
A.只有一条\hfill B.只有两条\hfill C.只有三条\hfill D.不存在\\
\textbf{解:}
$$\text{曲线切向量}(1,-2t,3t^2)$$
$$\text{平面法向量}(27,54,27)$$
\begin{align*}
	(27,54,27)\cdot (1,-2t,3t^2) &=0\\
	27(1-4t+3t^2)&=0\\
	(t-1)(3t-1)&=0
\end{align*}
$$\Rightarrow t=1\text{或}\frac{1}{3}$$
$$t=1,(1,-1,1)\text{不在平面内}$$
$$t=\frac{1}{3},(\frac{1}{3},-\frac{1}{9},\frac{1}{27})\text{在平面内,舍去}$$
$$\Rightarrow \text{选A}$$
\newpage
\noindent
3.微分方程$y''+y'+y=e^{-\frac{1}{2}x}\sin \frac{\sqrt{3}}{2}x$的一个特解应具有形式(其中a,b为常数)(\quad).\\
A.$e^{-\frac{1}{2}x}\left(a\sin \frac{\sqrt{3}}{2}x + bx\cos\frac{\sqrt{3}}{2}x \right)$\hfill
B.$e^{-\frac{1}{2}x}\left(a\cos \frac{\sqrt{3}}{2}x + b\sin\frac{\sqrt{3}}{2}x \right)$\\
C.$xe^{-\frac{1}{2}x}\left(a\cos \frac{\sqrt{3}}{2}x + b\sin\frac{\sqrt{3}}{2}x \right)$\hfill
D.$e^{-\frac{1}{2}x}\left(a\cos \frac{\sqrt{3}}{2}x + bx\sin\frac{\sqrt{3}}{2}x \right)$\\
\textbf{解:}
\begin{align*}
    \lambda^2 + \lambda +1 &=0\\
    \lambda&=\frac{-1 \pm \sqrt{3}i}{2}
\end{align*}
%$$Y=e^{-\frac{1}{2}x} \left(C_1 \cos \frac{\sqrt{3}}{2}x + C_2 \sin \frac{\sqrt{3}}{2}x \right)$$
$$y^{*}=xe^{-\frac{1}{2}x}\left(A\cos \frac{\sqrt{3}}{2}x+B\sin \frac{\sqrt{3}}{2}x\right)$$
$$\Rightarrow \text{选C}$$
4.利用变量代换$u=x,v=\frac{y}{x}$,可将方程$x\frac{\partial z}{\partial x}+y\frac{\partial z}{\partial y}=z$化为新方程(\quad).\\
A.$u\frac{\partial z}{\partial u}=z$\hfill B.$v\frac{\partial z}{\partial v}=z$\hfill C.$u\frac{\partial z}{\partial v}=z $\hfill  D.$v\frac{\partial z}{\partial u}=z$\\
\textbf{解:}
$$\frac{\partial z}{\partial x}=\frac{\partial z}{\partial u}-\frac{y}{x^2}\frac{\partial z}{\partial v}$$
$$\frac{\partial z}{\partial y}=\frac{1}{x}\frac{\partial z}{\partial v}$$
$$x=u,y=xv=uv$$
\begin{align*}
    x\frac{\partial z}{\partial x}+y\frac{\partial z}{\partial y}&=z\\
    u\frac{\partial z}{\partial u}-u\frac{uv}{u^2}\frac{\partial z}{\partial v}+uv\frac{1}{u}\frac{\partial z}{\partial v}&=z\\
    u\frac{\partial z}{\partial u}=z
\end{align*}
$$\Rightarrow \text{选A}$$
5.设函数$f(t)$ 连续,区域$D=\{\left. \left(x,y\right) \right| x^2+y^2\le 2y \}$,则$\iint\limits_{D}f(xy)\mathrm{d}x\mathrm{d}y=$(\quad).\\
A.$\int_{-1}^{1}\mathrm{d}x\int_{-\sqrt{1-x^2}}^{\sqrt{1+x^2}}f(xy)\mathrm{d}y$\hfill
B.$\int_{0}^{2}\mathrm{d}y\int_{0}^{\sqrt{2y-y^2}}f(xy)\mathrm{d}x$\\
C.$\int_{0}^{\pi}\mathrm{d}\theta\int_{0}^{2\sin \theta}f(r^2\sin\theta\cos\theta)\mathrm{d}r$\hfill
C.$\int_{0}^{\pi}\mathrm{d}\theta\int_{0}^{2\sin \theta}f(r^2\sin\theta\cos\theta)r\mathrm{d}r$\\
\textbf{解:}
\begin{align*}
    &\quad \iint\limits_{D}f(xy)\mathrm{d}x\mathrm{d}y\\
    &= \int_{-1}^{1}\mathrm{d}x\int_{1-\sqrt{1-x^2}}^{1+\sqrt{1+x^2}}f(xy)\mathrm{d}y\\
    &= \int_{0}^{2}\mathrm{d}y\int_{-\sqrt{2y-y^2}}^{\sqrt{2y-y^2}}f(xy)\mathrm{d}x\\
    &= \int_{0}^{\pi}\mathrm{d}\theta\int_{0}^{2\sin \theta}f(r^2\sin\theta\cos\theta)r\mathrm{d}r
\end{align*}
$$\Rightarrow \text{选D}$$
6.设$A,B$是$n$阶实对称可逆矩阵,则存在$n$阶可逆矩阵$P$,使下列关系式\\
\centerline{
    \ding{172}$PA=B;$ 
    \ding{173}$P^{-1}ABP=BA;$
    \ding{174}$P^{-1}AP=B;$ 
    \ding{175}$P^T A^2 P =B^2$.}
成立的个数为(\quad ).\\
A.1\hfill B.2\hfill C.3 \hfill D.4\\
\textbf{解:}
$$P=BA^{-1}\text{可逆}$$
$$\Rightarrow PA=BA^{-1}A=B$$
$$\Rightarrow  \text{\ding{172}成立}$$
$$P=A\text{可逆}$$
$$\Rightarrow P^{-1}ABP=A^{-1}ABA=BA$$
$$\Rightarrow  \text{\ding{173}成立}$$
$$A=\begin{pmatrix}
    1&0&0\\
    0&1&0\\
    0&0&1\\
\end{pmatrix}$$
$$B=\begin{pmatrix}
    1&0&0\\
    0&3&0\\
    0&0&5\\
\end{pmatrix}$$
$$\text{特征值不相同}\Rightarrow A\text{不相似于}B \Rightarrow \text{\ding{174}不成立}$$
$$A,B\text{可逆}\Rightarrow \text{惯性系数加起来=n,平方后只有正的惯性系数,\ding{175}成立}$$
$$\Rightarrow \text{选C}$$
\newpage
\noindent
7.$\alpha_1,\alpha_2,\alpha_3,\alpha_4,\alpha_5$均是4维列向量,记$A=\left(\alpha_1,\alpha_2,\alpha_3,\alpha_4 \right),B=\left(\alpha_1,\alpha_2,\alpha_3,\alpha_4,\alpha_5 \right)$.\\
已知方程组$Ax=\alpha_5$有通解$k\left(1,-1,2,0\right)^T+\left(2,1,0,1\right)^T$,其中$k$是任意常数,则下列\\
向量不是方程组$Bx=0$的解的是(\quad).\\
A.$\left(1,-2,-2,0,-1\right)^T$\hfill
B.$\left(0,3,-4,1,-1\right)^T$\hfill
C.$\left(2,1,0,1,-1\right)^T$\hfill
D.$\left(3,0,2,1,-1\right)^T$\\
\textbf{解:}
$$A\begin{pmatrix}
    2\\
    1\\
    0\\
    1\\
\end{pmatrix} =\alpha_5$$
$$\Rightarrow 2\alpha_1+\alpha_2+\alpha_4=\alpha_5$$
$$B\begin{pmatrix}
    1\\
    -2\\
    -2\\
    0\\
    -1\\
\end{pmatrix} =\alpha_1-2\alpha_2-2\alpha_3-\alpha_5=0$$
$$\Rightarrow A\begin{pmatrix}
    1\\
    -2\\
    -2\\
    0\\
\end{pmatrix} \neq \alpha_5$$
$$\Rightarrow \text{选A}$$
8.二次型$f\left(x_1,x_2,x_3\right)=x^T \begin{bmatrix}
    1&0&2\\
    -2&-3&2\\
    0&-8&0\\
\end{bmatrix}x$的秩为(\quad).\\
A.0\hfill B.1\hfill C.2\hfill D.3\\
\textbf{解:}
$$f\left(x_1,x_2,x_3\right)=x^T \begin{bmatrix}
    1&-1&1\\
    -1&-3&-3\\
    1&-3&0\\
\end{bmatrix}x$$
$$\begin{bmatrix}
    1&-1&1\\
    -1&-3&-3\\
    1&-3&0\\
\end{bmatrix} \rightarrow \begin{bmatrix}
    1&-1&1\\
    0&-4&-2\\
    0&-2&-1\\
\end{bmatrix} \rightarrow \begin{bmatrix}
    1&-1&1\\
    0&-2&-1\\
    0&0&0\\
\end{bmatrix}$$
$$\Rightarrow \text{选C}$$
\newpage
\noindent
9.设随机变量$X\sim B\left(4,\frac{2}{3}\right),Y\sim B\left(8,\frac{4}{5}\right)$,且相关系数$\rho_{XY}=-1$,则(\quad).\\
A.$P\left\{Y=-1.2X-9.6\right\}=1$\hfill
B.$P\left\{Y=-1.2X+9.6\right\}=1$\\
C.$P\left\{Y=-1.2X-3.2\right\}=1$\hfill
D.$P\left\{Y=-1.2X+3.2\right\}=1$\\
\textbf{解:}
  设$Y=aX+b$
  $$E(X)=\frac{8}{3},D(X)=\frac{8}{9},E(Y)=\frac{32}{5},D(Y)=\frac{32}{25}$$
  \begin{align*}
    Cov(X,Y)&=\rho_{XY}\sqrt{D(X)}\sqrt{D(Y)}\\
            &=-\frac{16}{15}
  \end{align*}
  $$a=\frac{Cov(X,Y)}{D(X)}=-\frac{6}{5}=-1.2$$
  $$b=E(Y)-E(X)\frac{Cov(X,Y)}{D(X)}=\frac{48}{5}=9.6$$
  $$\Rightarrow \text{选B}$$
10.设总体$X\sim N\left(a,\sigma^2 \right),Y\sim N(b,\sigma^2)$,且相互独立,分别从$X$和$Y$中各抽取容\\
量为9和10的简单随机样本,记他们的方差分别为$S_X^2$和$S_Y^2$,并记
$S_{12}^2=\frac{1}{2}\left(S_X^2+S_Y^2\right)$\\和$S_{XY}^2=\frac{1}{18}\left(8S_X^2+10S_Y^2\right)$,
则这四个统计量$S_X^2,S_Y^2,S_{12}^2,S_{XY}^2$中,方差最小的是(\quad).\\
A.$S_X^2$\hfill B.$S_Y^2$ \hfill C.$S_{12}^2$ \hfill D.$S_{XY}^2$\\
\textbf{解:}
$$\frac{(9-1)S_X^2}{\sigma^2} \sim \chi(9-1)$$
\begin{align*}
    D(\chi(9-1))&=D(\frac{(9-1)S_X^2}{\sigma^2})\\
    2(9-1)&=\frac{8^2}{\sigma^4}D(S_X^2)\\
    D(S_X^2)&=\frac{\sigma^4}{4}
\end{align*}
$$\frac{(10-1)S_Y^2}{\sigma^2} \sim \chi(10-1)$$
\begin{align*}
    D(\chi(10-1))&=D(\frac{(10-1)S_Y^2}{\sigma^2})\\
    2(10-1)&=\frac{9^2}{\sigma^4}D(S_Y^2)\\
    D(S_Y^2)&=\frac{2\sigma^4}{9}
\end{align*}
$$D(S_{12}^2)=\frac{1}{4}(\frac{\sigma^4}{4}+\frac{2\sigma^4}{9})=\frac{17}{144}\sigma^4$$
$$D(S_{XY}^2)=\frac{1}{18^2}(8^2\frac{\sigma^4}{4}+10^2\frac{2\sigma^4}{9})=\frac{86}{729}\sigma^4$$
$$D(S_X^2)>D(S_Y^2)>D(S_{12}^2)>D(S_{XY}^2)\Rightarrow \text{选D}$$
\noindent
\textbf{二、填空题(每题6分共10题,共计60分}\\
11.设$x>0$且$x\neq 1$,则$\lim\limits_{n\to \infty }n^2\left(\sqrt[n-1]{x}-\sqrt[n]{x}\right)=$\underline{\hbox to 20mm{}}.\\
\textbf{解:}
\begin{align*}
    \lim\limits_{n\to\infty} n^2\left(\sqrt[n-1]{x}-\sqrt[n]{x}\right)
    &=\lim\limits_{n\to \infty} n^2 \left(e^{\frac{\ln x}{n-1}}-e^{\frac{\ln x}{n}}\right)\\
    &=\lim\limits_{n\to\infty} n^2 e^{\frac{\ln x}{n}} \left(e^{\frac{\ln x}{n(n-1)}}-1\right)\\
    &=\lim\limits_{n\to\infty} \frac{n^2\ln x}{n(n-1)}\\
    &=\ln x
\end{align*}
12.设函数$f(x)$ 在$\left[1,+\infty\right)$ 上连续$\int_{1}^{+\infty} f(x)\mathrm{d}x $ 收敛,则满足\\
$f(x)=\frac{\ln x}{(1+x)^2}+\frac{1+x^2}{1+x^4}\int_{1}^{+\infty}f(x)\mathrm{d}x$,则$\int_{1}^{+\infty} f(x)\mathrm{d}x=$\underline{\hbox to 20mm{}}.\\
\textbf{解:}
因为收敛,设$A=\int_{1}^{+\infty} f(x)\mathrm{d}x$
\begin{align*}
    f(x)&=\frac{\ln x}{(1+x)^2}+\frac{1+x^2}{1+x^4}\int_{1}^{+\infty}f(x)\mathrm{d}x\\
    A &=\int_{1}^{+\infty} \frac{\ln x}{(1+x)^2} \mathrm{d}x +A\int_{1}^{+\infty} \frac{1+x^2}{1+x^4}\mathrm{d}x\\
    A&=\left. -\frac{\ln x}{1+x} \right|_{1}^{+\infty} +\int_{1}^{+\infty} \frac{1}{x(1+x)}\mathrm{d} x +A\int_{0}^{1}\frac{1+x^2}{1+x^4}\mathrm{d} x \\
    A&=\left. \ln \frac{x}{1+x}\right|_{1}^{+\infty} +\frac{A}{2}\int_{0}^{+\infty} \frac{1+x^2}{1+x^4}\mathrm{d} x\\
    A&=\ln 2 +\frac{A\pi i}{2}\left[Res\left[\frac{1+z^2}{1+z^4},e^{i\frac{\pi}{4}}\right] +Res\left[\frac{1+z^2}{1+z^4},e^{i\frac{3\pi}{4}}\right]  \right]\\
    A&=\ln 2 +\frac{A\pi i}{2}\left[-\frac{e^{i\frac{ \pi }{4}}+e^{i\frac{ 3\pi }{4}}}{4}-\frac{e^{i\frac{ \pi }{4}}+e^{i\frac{ 3\pi }{4}}}{4} \right]\\
    A&=\ln 2 +\frac{\sqrt{2}\pi}{4}A\\
    A&=\frac{-\ln 2}{\frac{\sqrt{2}\pi}{4}-1}\\
    A&=\frac{4\ln 2}{4-\sqrt{2}\pi}
\end{align*}
\newpage
\noindent
13.设函数$f(x,y)$在区域$D=\{x^2+y^2\le 1 \}$上连续,且满足\\
	$(f''_{xx}+f''_{yy})e^{x^2+y^2}=1$,则$\iint \limits_D xf'_{x}+yf'_{y}\mathrm{d}\sigma$\underline{\hbox to 20mm{}}.\\
\textbf{解:}

\begin{align*}
    \iint\limits_{D} xf'_{x}+yf'_{y}\mathrm{d}\sigma &=\int_{0}^{2\pi} \mathrm{d}\theta \int_{0}^{1}(\rho\cos \theta f'_{x}+\rho\sin \theta f'_{y})\rho \mathrm{d}\rho\\
    &=\int_{0}^{1} \rho\mathrm{d}\rho \int_{0}^{2\pi} (\frac{\partial y}{\partial \theta}f'_{x}-\frac{\partial x}{\partial \theta}f'_{y})\mathrm{d}\theta\\
    &=\int_{0}^{1}\rho\mathrm{d}\rho\int_{0}^{2\pi}-f'_{y}\mathrm{d}x+f'_{x}\mathrm{d}y\\
    &=\int_{0}^{1}\rho\mathrm{d}\rho \iint\limits_{x^2+y^2\le \rho} (f''_{xx}+f''_{yy})d\sigma\\
    &=\int_{0}^{1}\rho\mathrm{d}\rho \iint\limits_{x^2+y^2\le \rho} e^{-(x^2+y^2)}d\sigma\\
    &=\int_{0}^{1}\rho \mathrm{d}\rho \int_{0}^{2\pi}\mathrm{d} \theta \int_{0}^{\rho} e^{-r^2} r\mathrm{d}r\\
    &=\int_{0}^{1}\rho \mathrm{d}\rho \int_{0}^{2\pi} \left. \frac{e^{-r^2}}{-2} \right|_{0}^{\rho} \mathrm{d} \theta\\
    &=\int_{0}^{1} \rho \mathrm{d}\rho \int_{0}^{2\pi}\frac{1-e^{-\rho^2}}{2}\mathrm{d}\theta \\
    &=\pi\int_{0}^{1} (1-e^{-\rho^2})\rho \mathrm{d}\rho\\
    &=\left.\pi(\frac{\rho^2}{2}+\frac{e^{-\rho^2}}{2})\right|_{0}^{1}\\
    &=\frac{\pi}{2e}
\end{align*}
14.设$z=f(\sqrt{x^2+y^2})$,其中$f(u)$有二阶连续导数,$f(0)=f'(0)=0$,\\
且$\frac{\partial^2 z}{\partial x^2}+\frac{\partial^2 z}{\partial y^2}-\frac{1}{x}\frac{\partial z}{\partial x}=z+\sqrt{x^2+y^2}$,求$f(u)=$\underline{\hbox to 20mm{}}.\\
\textbf{解:}
令$u=\sqrt{x^2+y^2}$
$$\frac{\partial z}{\partial x}=\frac{\partial z}{\partial u}\frac{\partial u}{\partial x} = f'(u)\frac{x}{\sqrt{x^2+y^2}}$$
$$\frac{\partial z}{\partial y}=\frac{\partial z}{\partial u}\frac{\partial u}{\partial y} = f'(u)\frac{y}{\sqrt{x^2+y^2}}$$
\begin{align*}
\frac{\partial^2 z}{\partial x^2}&=f''(u) \frac{x^2}{x^2+y^2}+f'(u)\frac{\sqrt{x^2+y^2} -\frac{x^2}{\sqrt{x^2+y^2}}}{x^2+y^2}\\
&=f''(u)\frac{x^2}{x^2+y^2}+f'(u)\frac{y^2}{(x^2+y^2)^{\frac{3}{2}}}
\end{align*}
$$\frac{\partial^2 z}{\partial y^2}=f''(u)\frac{y^2}{x^2+y^2}+f'(u)\frac{x^2}{(x^2+y^2)^{\frac{3}{2}}}$$
\begin{align*}
\frac{\partial^2 z}{\partial x^2}+\frac{\partial^2 z}{\partial y^2}-\frac{1}{x}\frac{\partial z}{\partial x}&=z+\sqrt{x^2+y^2}\\
f''(u)-f(u)&=u
\end{align*}
$$\lambda^2-\lambda=0$$
$$\lambda = \pm 1$$
$$ F(u)=C_{1} e^{u}+C_{2}e^{-u}$$
$$\text{设}f^{*}(u)=au+b$$
$$\text{代入得},a=-1,b=0$$
$$f(u)=C_{1} e^{u}+C_{2}e^{-u}-u$$
$$f(0)=f'(0)=0\Rightarrow C_{1}=\frac{1}{2},C_{2}=-\frac{1}{2}$$
$$f(u)=\frac{1}{2}e^{u}-\frac{1}{2}e^{-u} -u$$
15.求直线$L: \frac{x-3}{2} = \frac{y-1}{3} = z+1$绕直线$L_{1}:\begin{cases} 
	x=2\\
	y=3		
	\end{cases}$ 旋转一周所成的曲面方程\underline{\hbox to 20mm{}}.\\
\textbf{解:}
$L$的参数方程$\begin{cases}
x=3+2t\\
y=1+3t\\
z=-1+t
\end{cases}$\\
设$(x_{0},y_{0},z_{0})$为$L$上的一点,$(x,y,z)$为$(x_{0},y_{0},z_{0})$绕$L_{1}$旋转后的在曲面方程上的一点
$$\begin{cases}
(x-2)^2+(y-3)^2+z^2=(x_{0}-2)^2+(y_{0}-3)^2+z_{0}^2\\
z=z_{0}\\
x_{0}=3+2t\\
y_{0}=1+3t\\
t=z+1

\end{cases}$$
$$\Rightarrow x^2+y^2-13z^2-4x-6y-18z+3=0$$
16.设$A=E+\alpha \beta^T$,其中$\alpha,\beta$均为n维列向量,$\alpha^T\beta=3$,则$\left| A+2E\right|=$\underline{\hbox to 20mm{}}\\
\textbf{解:}
$$\text{设}\alpha \text{是}n\text{维列向量}$$
$$\alpha^T \beta=3 \Rightarrow \alpha\neq=0 ,\beta \neq=0\Rightarrow r(\alpha\beta^T)=1$$
$$\alpha \beta ^T \alpha=3\alpha$$
$$\alpha \beta ^T \text{特征值为}3,n-1\text{个} 0$$
$$A+2E=3E+\alpha \beta ^T \text{特征值为}6,n-1\text{个} 3$$
$$\left| A+2E\right|=2\cdot 3^n$$
17.设$A$是3阶实对称矩阵,$\lambda =5$是$A$的二重特征值,对应的特征向量为\\
$\xi_1=\left[1,-1,2\right]^T,\xi_2=\left[1,2,1\right]^T$,则二次型$f(x_1,x_2,x_3)=x^T A x$在$x_0=\left[1,5,0\right]^T$的值\\
$f(1,5,0)=$\underline{\hbox to 20mm{}}.\\
\textbf{解:}
设$\lambda_1=\lambda_2=5,\lambda_3$为$A$的特征值,对应的特征向量为$\xi_1,\xi_2,\xi_3$
$$\xi_1^T \xi_3=0,\xi_2^T \xi_3=0$$
$$x_0=-\xi_1+2\xi_2=\left(\xi_1,\xi_2,\xi_3\right) \left(-1,2,0\right)^T$$
$$\Rightarrow Ax_0=5Ax_0$$
\begin{align*}
    f(1,5,0)&=x_0^T A x_0\\
    &=(-1,2,0)\begin{pmatrix}
        \xi_1^T\\
        \xi_2^T\\
        \xi_3^T
    \end{pmatrix}A (\xi_1,\xi_2,\xi_3)\begin{pmatrix}
        -1\\
        2\\
        0
    \end{pmatrix}\\
    &=(-\xi_1+2\xi_2)^T(5\xi_1,5\xi_2,5 \xi_3)\begin{pmatrix}
        -1\\
        2\\
        0
    \end{pmatrix}\\
    &=5(-\xi_1+2\xi_2)^T(-\xi_1+2\xi_2)\\
    &=130
\end{align*}
18.设$A$是三阶矩阵,$b=\left[9,18,-18\right]^T$,方程组$Ax=b$有通解\\
$k_1\left[-2,1,0\right]^T+k_2\left[2,0,1\right]^T+\left[1,2,-2\right]^T$,
其中$k_1,k_2$是任意常数,则$A^{100}=$\underline{\hbox to 20mm{}}\\
\textbf{解:}
设$\alpha_1=\begin{pmatrix}
    -2\\
    1\\
    0\\
\end{pmatrix},\alpha_2=\begin{pmatrix}
    2\\
    0\\
    1\\
\end{pmatrix},\alpha_3=\begin{pmatrix}
    1\\
    2\\
    -2\\
\end{pmatrix}$
$$\alpha_1,\alpha_2,\alpha_3\text{线性无关}$$
$$A\alpha_1 =0\alpha_1$$
$$A\alpha_2=0\alpha_2$$
$$A\alpha_3=b=9\alpha_3$$
$$\text{所以}A\text{特征值为}0,0,9$$
设$P=(\alpha_1,\alpha_2,\alpha_3)$
$$P^{-1}AP=diag(0,0,9)$$
\begin{align*}
    \begin{pmatrix}
        -2&2&1&1&0&0\\
        1&0&2&0&1&0\\
        0&1&-2&0&0&1\\
    \end{pmatrix} & \rightarrow \begin{pmatrix}
        1&0&2&0&1&0\\
        0&1&-2&0&0&1\\
        -2&2&1&1&0&0\\
    \end{pmatrix}\\
    &\rightarrow \begin{pmatrix}
        1&0&2&0&1&0\\
        0&1&-2&0&0&1\\
        0&2&5&1&2&0\\
    \end{pmatrix}\\
    &\rightarrow \begin{pmatrix}
        1&0&2&0&1&0\\
        0&1&-2&0&0&1\\
        0&0&9&1&2&-2\\
    \end{pmatrix}\\
    &\rightarrow \begin{pmatrix}
        1&0&2&0&1&0\\
        0&1&-2&0&0&1\\
        0&0&1&\frac{1}{9}&\frac{2}{9}&-\frac{2}{9}\\
    \end{pmatrix}\\
    &\rightarrow \begin{pmatrix}
        1&0&0&-\frac{2}{9}&\frac{5}{9}&\frac{4}{9}\\
        0&1&0&\frac{2}{9}&\frac{4}{9}&\frac{5}{9}\\
        0&0&1&\frac{1}{9}&\frac{2}{9}&-\frac{2}{9}\\
    \end{pmatrix}
\end{align*}
$$P^{-1}=\frac{1}{9}\begin{pmatrix}
    -2&5&4\\
    2&4&5\\
    1&2&-2\\
\end{pmatrix}$$
\begin{align*}
    A^{100}&=(P\ diag(0,0,9)\ P^{-1}))^{100}\\
    &=9^{100} P\ diag(0,0,1)P^{-1}\\
    &=9^{100}\begin{pmatrix}
        0&0&1\\
        0&0&2\\
        0&0&-2\\
    \end{pmatrix}\begin{pmatrix}
        -2&5&4\\
        2&4&5\\
        1&2&-2\\
    \end{pmatrix}\\
    &=9^{100}\begin{pmatrix}
        1&2&-2\\
        2&4&-4\\
        -2&-4&4\\
    \end{pmatrix}
\end{align*}
19.甲、乙两人轮流投篮,甲先投,甲每轮只投篮一次,而乙每轮投篮两次,先\\
投中者为胜.已知甲、乙每次投篮命中率分别为$p,0.5$,且每人命中与否相互独\\
立,若甲、乙两人胜率相同,则$p=$\underline{\hbox to 20mm{}}\\
\textbf{解:}
设事件$A$为甲获胜
\begin{align*}
    P(A)&=P(\overline{A} )\\
    p+(1-p)\frac{1}{2}p+(1-p)\frac{1}{2}(1-p)\frac{1}{2}p+\cdots &= (1-p)\frac{1}{2}+(1-p)\frac{1}{2}(1-p)\frac{1}{2}+(1-p)\frac{1}{2}(1-p)\frac{1}{2}(1-p)\frac{1}{2}+\cdots \\
    p\sum_{n=0}^{\infty}((1-P)\frac{1}{2})^n&=(1-p)\frac{1}{2}\sum_{n=0}^{\infty}((1-P)\frac{1}{2})^n\\
    p&=\frac{1}{3}
\end{align*}
20.已知随机变量$X$在$\left(1,2\right)$上服从均匀分布,在$X=x$条件下$Y$服从参数为$x$\\
的指数分布,则$E(XY^2)=$\underline{\hbox to 20mm{}}\\
\textbf{解:} 
\\
$f_{\left. Y\right| X} (\left. y \right| x)=\begin{cases}
    xe^{-xy},y>0\\
    0,y\le 0
\end{cases}$\\
$f(x,y)=f_{X}(x) f_{\left. Y\right| X} (\left. y \right| x) =\begin{cases}
    xe^{-xy},1<x<2 ,y>0\\
    0,\text{其他}
\end{cases}$
\begin{align*}
    E(XY^2)
    &=\int_{1}^{2}\mathrm{d}x\int_{0}^{+\infty} xy^2 xe^{-xy} \mathrm{d}y\\
    &=\int_{1}^{2}x^2\mathrm{d}x\int_{0}^{+\infty} y^2 e^{-xy} \mathrm{d}y\\
    &=\int_{1}^{2}\frac{\Gamma(3)}{x}\mathrm{d}x\\
    &=\left. 2\ln x\right|_{1}^{2}\\
    &=2\ln 2
\end{align*}
\textbf{三、解答题(共6小题,共计130分)}\\
解答应写出文字说明、证明过程或演算步骤。\\
21.(20分)求下列极限.\\
(1)$\lim\limits_{n\to\infty}(1+\sin(\pi \sqrt{4n^2+2}))^n$\\
(2)$\lim\limits_{x\to 0}\left[\frac{\arctan (1-e^x)}{\sqrt{1+4x^2}}+(\frac{\arctan x}{\tan x})^\frac{3}{\sin^2 x}\right]$\\
(3)$\lim\limits_{x \to 0} \frac{\tan(\tan x)-\sin(\sin x)}{x-\sin x}$\\
(4)$\lim\limits_{x\to 0^{+}} \frac{x^x -(\sin x)^x}{x^2\ln(1+x)}$\\
\textbf{解:}\\
(1)\\
\begin{align*}
	\lim \limits_{n\to\infty} (1+\sin(\pi\sqrt{4n^2+2}))^{n}&=\lim \limits_{n\to\infty} (1+\sin(\pi\sqrt{4n^2+2}-2n\pi))^{n}\\
	&=\lim \limits_{n\to\infty} (1+\sin(\frac{2\pi}{\sqrt{4n^2+2}+2n}))^{n}\\
	&=exp(\lim\limits_{n\to\infty} n\sin(\frac{2\pi}{\sqrt{4n^2+2}+2n}))\\
	&=exp(\lim\limits_{n\to\infty} \frac{2n\pi}{\sqrt{4n^2+2}+2n})\\
	&=e^{\frac{2\pi}{4}}\\
	&=e^{\frac{\pi}{2}}
\end{align*}
(2)\\
\begin{align*}
    &\quad \lim\limits_{x\to 0}\left[\frac{\arctan (1-e^x)}{\sqrt{1+4x^2}}+(\frac{\arctan x}{\tan x})^\frac{3}{\sin^2 x}\right]\\
    &=\frac{0}{1}+ \lim\limits_{x\to 0}(1+\frac{\arctan x-\tan x}{\tan x})^{\frac{3}{\sin^2 x}}\\
    &=exp(\lim\limits_{x\to 0} \frac{3(\arctan x -\tan x)}{\tan x \sin^2 x})\\
    &=exp(\lim\limits_{x\to 0} \frac{3(-\frac{2}{3}x^3)}{x^3})\\
    &=\frac{1}{e^2}
\end{align*}
(3)\\
\begin{align*}
	&\quad\lim\limits_{x \to 0} \frac{\tan(\tan x)-\sin(\sin x)}{x-\sin x}\\&= \lim \limits_{x \to 0} \frac{\tan(\tan x) -\tan x +\tan x -\sin x + \sin x -\sin(\sin x)}{x-\sin x}\\
	&=\lim \limits_{x\to 0}\frac{\frac{1}{3}\tan^3 x +\frac{1}{2}x^3 +\frac{1}{6}\sin^3 x}{\frac{1}{6}x^3}\\
	&=6
\end{align*}
(4)\\
\begin{align*}
	\lim\limits_{x\to 0^{+}} \frac{x^x -(\sin x)^x}{x^2\ln(1+x)}&=
	\lim\limits_{x\to 0^{+}} x^x \lim\limits_{x\to 0^{+}} \frac{1-(\frac{\sin x}{x})^x}{x^3}\\
	&=\lim\limits_{x\to 0^{+}} \frac{1-e^{x\ln(1+\frac{\sin x - x}{x})}}{x^3}\\
	&=\lim\limits_{x\to 0^{+}} \frac{x-\sin x}{x^3}\\
	&=\frac{1}{6}
\end{align*}
22.(20分)设$a$与$b$都是常数,且$b>a>0$.\\
(1)写出$yOz$上的圆$(y-b)^2+z^2=a^2$绕$Oz$轴旋转一周生成的环面$\Sigma$的方程;\\
(2)记$\Sigma$所围成的空间区域为$\Omega$,计算三重积分$\iiint\limits_{\Omega}(x+y)^2\mathrm{d}v$.\\
\textbf{解:}\\
(1)\\
$(\sqrt{x^2+y^2}-b)^2+z^2=a^2$\\
(2)\\
$z=\pm \sqrt{a^2-(r-b)^2}$
\begin{align*}
    &\quad \iiint\limits_{\Omega}(x+y)^2\mathrm{d}v\\
    &=\int_{0}^{2\pi} \mathrm{d} \theta \int_{b-a}^{b+a} r\mathrm{d}r\int_{-\sqrt{a^2-(r-b)^2}}^{\sqrt{a^2-(r-b)^2}} (r\cos\theta+r\sin \theta)^2\mathrm{d}z\\
    &=\int_{0}^{2\pi}(1+2\sin \theta\cos \theta)\mathrm{d}\theta \int_{b-a}^{b+a} 2r^3\sqrt{a^2-(r-b)^2}\mathrm{d} r\\
    &=2\pi \int_{-a}^{a}2(u+b)^3\sqrt{a^2-u^2}\mathrm{d}u\\
    &=4\pi\int_{-a}^{a}(u^3+3u^2b+3ub^2+b^3)\sqrt{a^2-u^2}\mathrm{d}u\\
    &=8\pi\int_{0}^{a}(3u^2b+b^3)\sqrt{a^2-u^2}\mathrm{d}u\\
    &=8\pi(\frac{b^3\pi a^2}{4}+\int_{0}^{\frac{\pi}{2}}3a^2 b\sin^2 t\ a\cos t\ a\cos t\ \mathrm{d} t)\\
    &=2a^2 b^3\pi^2+8\pi \int_{0}^{\frac{\pi}{2}} 3a^4b \sin^2 t\ \cos^2 t \mathrm{d}t\\
    &=2a^2 b^3\pi^2+8\pi \int_{0}^{\frac{\pi}{2}} 3a^4b (\sin^2 t-\sin^4 t)\mathrm{d}t\\
    &=2a^2 b^3\pi^2+8\pi\ 3a^4b(\frac{1}{2}\frac{\pi}{2}-\frac{3}{4}\frac{1}{2}\frac{\pi}{2})\\
    &=2a^2 b^3\pi^2+\frac{3a^4 b \pi^2}{2}
\end{align*}
23.(20分)设函数$f(x)$在区间$\left[a,b\right]$上具有连续导数,$f'(x)>0$,且$a\le f(x)\le b$.\\
求证:\\
(1)对任意$x_1,x_2 \in (a,b)$,存在$c\in(a,b)$ ,使得$f'(c)=\sqrt{f'(x_1)f'(x_2)}$;\\
(2)存在$\xi\in(a,b)$使得$f\left[f(a)\right]-f\left[f(b)\right]=\left[f'(\xi)\right]^2(a-b)$.\\
\textbf{解:}
(1)\\
不妨设$x_1\le x_2$\\
$min(f'(x_1),f'(x_2))\le \sqrt{f'(x_1)f'(x_2)} \le max(f'(x_1),f'(x_2))$\\
由介值定理$\exists c\in\left[x_1,x_2\right]\subset (a,b)$,使得$f'(c)=\sqrt{f'(x_1)f'(x_2)}$\\
(2)\\
由拉格朗日中值定理
\begin{align*}
    f\left[f(a)\right]-f\left[f(b)\right]&=f'(x_1) (f(a)-f(b)) \left(x_1\text{介于}f(a),f(b)\text{之间,即}x_1\in(a,b)\right)\\
    f\left[f(a)\right]-f\left[f(b)\right]&=f'(x_1)f'(x_2)(a-b)\left(x_1,x_2 \in (a,b)\right)\\
    f\left[f(a)\right]-f\left[f(b)\right]&=\left[f'(\xi)\right]^2(a-b)\left(\xi \in (a,b)\right)
\end{align*}
24.(20分)设$\begin{cases}
    x_n=x_{n-1}+2y_{n-1}\\
    y_n=4x_{n-1}+3y_{n-1}
\end{cases},(n=1,2,3,\cdots )$,且$x_0=2,y_0=1$,求$x_{2077}$.\\
\textbf{解:}
$$\begin{pmatrix}
    x_n\\
    y_n\\
\end{pmatrix}=\begin{pmatrix}
    1&2\\
    4&3\\
\end{pmatrix}\begin{pmatrix}
    x_{n-1}\\
    y_{n-1}\\
\end{pmatrix}=\begin{pmatrix}
    1&2\\
    4&3\\
\end{pmatrix}^n\begin{pmatrix}
    x_{0}\\
    y_{0}\\
\end{pmatrix}=\begin{pmatrix}
    1&2\\
    4&3\\
\end{pmatrix}^n\begin{pmatrix}
    2\\
    1\\
\end{pmatrix}$$
$$A=\begin{pmatrix}
    1&2\\
    4&3\\
\end{pmatrix}$$
设$\lambda_1,\lambda_2$为$A$的特征值\\
$$\left| \lambda E -A\right| =\begin{pmatrix}
    \lambda-1&-2\\
    -4&\lambda -3\\
\end{pmatrix}=\lambda^2-4\lambda -5=(\lambda-5)(\lambda +1)=0$$
$$\Rightarrow \lambda_1=5,\lambda_2=-1$$
设$\xi_1,\xi_2$分别为$\lambda_1,\lambda_2$对应的特征向量\\
$(A-5E)\xi_1=0\Rightarrow \xi_1=\begin{pmatrix}
    1\\
    2\\
\end{pmatrix}$\\
$(A+E)\xi_2=0 \Rightarrow \xi_2=\begin{pmatrix}
    1\\
    -1\\
\end{pmatrix}$\\
$$P=\begin{pmatrix}
    1&1\\
    2&-1\\
\end{pmatrix}$$
$$P^{-1}=\frac{1}{-1-2}\begin{pmatrix}
    -1&-1\\
    -2&1
\end{pmatrix}=\frac{1}{3}\begin{pmatrix}
    1&1\\
    2&-1
\end{pmatrix}$$
$$P^{-1}AP=diag(5,-1)$$
\begin{align*}
    A^{2077}&=(P\ diag(5,-1)\ P^{-1})^{2077}\\
    &=\frac{1}{3} \begin{pmatrix}
        1&1\\
        2&-1
    \end{pmatrix} \begin{pmatrix}
        5^{2077}&0\\
        0&-1
    \end{pmatrix}\begin{pmatrix}
        1&1\\
        2&-1
    \end{pmatrix}\\
    &=\frac{1}{3} \begin{pmatrix}
        5^{2077}&-1\\
        2\cdot 5^{2077}&1\\
    \end{pmatrix} \begin{pmatrix}
        1&1\\
        2&-1
    \end{pmatrix}\\
    &=\frac{1}{3}\begin{pmatrix}
        5^{2077}-2&5^{2077}+1\\
        2\cdot 5^{2077}+2&2\cdot 5^{2077}-1\\
    \end{pmatrix}
\end{align*}
$$\begin{pmatrix}
    x_{2077}\\
    y_{2077}\\
\end{pmatrix}=A^{2077}\begin{pmatrix}
    2\\
    1\\
\end{pmatrix}=\frac{1}{2}\begin{pmatrix}
    2\cdot 5^{2077} -4+5^{2077}+1\\
    4\cdot 5^{2077}+4+2\cdot 5^{2077}-1\\
\end{pmatrix}=\begin{pmatrix}
    5^{2077} -1\\
    2\cdot 5^{2077}+1\\
\end{pmatrix}$$
$$\Rightarrow x_{2077}=5^{2077} -1$$
25.(20分)设三阶矩阵$P=\left(\alpha_1,\alpha_2,\alpha_3 \right)$,其中$\alpha_1,\alpha_2$分别是三阶矩阵$A$对应\\
于特征值$-1$与$1$的特征向量,且$(A-E)\alpha_3-\alpha_2=0$.\\
(1)证明$P$可逆;\\
(2)计算$P^{-1}A^{*}P$.\\
\textbf{解:}\\
(1)\\
\begin{align*}
    (A-E)\alpha_3-\alpha_2&=0\\
    A\alpha_3=\alpha_2+\alpha_3
\end{align*}
$\alpha_1,\alpha_2$对应$A$的不同特征值的特征向量,所以线性无关\\
设$k_1 \alpha_1+k_2\alpha_2+k_3\alpha_3=0\cdots (1$)\\
$A(k_1 \alpha_1+k_2\alpha_2+k_3\alpha_3)=-k_1\alpha_1+k_2\alpha_2+k_3\alpha_2+k_3\alpha_3=0\cdots (2)$\\
(1)-(2)\\
$2k_1\alpha_1 -k_3\alpha_2=0$\\
因为$\alpha_1,\alpha_2$线性无关,所以$k_1=k_3=0$\\
$\Rightarrow k_2=0$\\
所以$\alpha_1,\alpha_2,\alpha_3$线性无关\\
所以$P$可逆\\
(2)\\
$$Q=\begin{pmatrix}
    -1&0&0\\
    0&1&1\\
    0&0&1\\
\end{pmatrix}$$
$$AP=PQ\Rightarrow P^{-1}AP=Q \Rightarrow A \sim Q$$
$$Q\text{特征值}-1,1,1$$
$$\left|A\right|=-1$$
$$A^{*}\text{特征值}1,-1,-1$$
$$P^{-1}A^{*}P=\begin{pmatrix}
    1&0&0\\
    0&-1&1\\
    0&0&-1\\
\end{pmatrix}$$
26.(30分)设随机变量$X$的概率密度为$f_{X}(x)=\begin{cases}
    2x,0<x<1\\
    0,\text{其他}\\
\end{cases}$,在给定\\
$X=x(0<x<1)$的条件下,随机变量$Y$在$(-x,x)$上服从均匀分布.\\
(1)求$P\left\{\left. \frac{1}{2}<X<\frac{3}{2} \right|Y=EY\right\}$;\\
(2)判断$X$与$Y$的独立性、相关性,并给出理由;\\
(3)令随机变量$Z=X-Y$,求$f_{Z}(z)$\\
\textbf{解:}
(1)\\
$$f_{\left. Y\right| X}(\left. y\right| x)=\begin{cases}
    \frac{1}{2x},-x< y < x,\\
    0,\text{其他}\\
\end{cases}$$
$$f(x,y)=f_{X}(x) f_{\left. Y\right| X}(\left. y\right| x)=\begin{cases}
    1,0<x<1,-x< y < x,\\
    0,\text{其他}
\end{cases}$$
$$f_{Y}(y)=\begin{cases}
    \int_{y}^{1}\mathrm{d}x,0< y < 1\\
    \int_{-y}^{1}\mathrm{d}x, -1< y <0\\
\end{cases}=\begin{cases}
    1-y,0<y<1\\
    1+y,-1<y<0\\
\end{cases}=1-\left|y\right| (-1<y<1)$$
$$EY=\int_{-1}^{1}y(1-\left|y\right|)\mathrm{d}y=0$$
\begin{align*}
    P\left\{\left. \frac{1}{2}<X<\frac{3}{2} \right|Y=EY\right\}
    &=P\left\{\left. \frac{1}{2}<X<\frac{3}{2} \right|Y=0\right\}\\
    &=\int_{\frac{1}{2}}^{\frac{3}{2}}\frac{f(x,y)}{f_{Y}(y)}\mathrm{d}x\\
    &=\int_{\frac{1}{2}}^{1}\mathrm{d}x\\
    &=\frac{1}{2}
\end{align*}
(2)\\
$$f_{X}(x)f_{Y}(y)\neq f(x,y),\Rightarrow \text{X,Y 不独立}$$
$$E(XY)=\int_{0}^{1}\mathrm{d} x \int_{-x}^{x} xy\mathrm{d}{y} =0$$
% $$E(X)=\int_{0}^{1}2x^2\mathrm{d}x=\frac{2}{3}$$
$$Cov(X,Y)=E(XY)-E(X)E(Y)=0\Rightarrow \text{不相关}$$
(3)\\
$$\begin{cases}
    z=x-y\\
    y=-x\\
\end{cases} \Rightarrow x=\frac{z}{2}$$
$$\text{当}0<z<2\text{时}$$
\begin{align*}
    F_{Z}(z)&=P(Z\le z)\\
    &=P(X-Y\le z)\\
    &=1-\int_{\frac{z}{2}}^{1}\mathrm{d}x\int_{-x}^{x-z}\mathrm{d}y\\
    &=1+\int_{\frac{z}{2}}^{1} (z-2x)\mathrm{d}x\\
    &=1+\left. (zx-x^2)\right|_{\frac{z}{2}}^{1}\\
    &=1+z-\frac{z^2}{2}-1+\frac{z^2}{4}\\
    &=-\frac{z^2}{4}+z
\end{align*}
$$\frac{\mathrm{d} F_{Z}(z)}{\mathrm{d}z}=1-\frac{z}{2}$$
$$f_Z(z)=\begin{cases}
    \frac{z}{2}-1,0<z<2\\
    0,\text{其他}
\end{cases}$$
\textbf{四、附加题(小作文,50分)}\\
将你对莉姆丝的爱用文字描绘出来,文体不限,诗歌更好,300字以上。\\
评分要素:修辞,剧情,情感,发病程度,好活赖活。\\
温馨提示:会进行枝网查重,查重率70\%以上不计分;名字写错,负分并记入\\
莉姆丝的假粉榜;写的特别好的按心情加分。
\newpage
粉丝牌\\
\includegraphics{fan_club.PNG}
\end{document}