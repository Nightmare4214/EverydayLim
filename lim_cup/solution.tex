\documentclass[a4paper]{article}
\usepackage{ctex}% 中文支持
\usepackage{geometry}% 用于页面设置
\usepackage[dvipsnames, svgnames, x11names]{xcolor} % 颜色支持
\usepackage[
colorlinks=true,
linkcolor=Navy,
urlcolor=Navy,
citecolor=Navy,
anchorcolor=Navy
]{hyperref}
\usepackage{tcolorbox}% 支持更好的文本框
\tcbuselibrary{skins, breakable}% 支持文本框跨页
\usepackage[english]{babel}% 载入美式英语断字模板
\usepackage{amsmath}
\usepackage{fancyhdr}
\usepackage{tasks}
\usepackage{esint}
\usepackage{pifont}
\usepackage{lmodern}
\pagestyle{fancy}
\lhead{}
\chead{lim杯}
\rhead{}
\lfoot{作者Nightmare4214}
\cfoot{\thepage}
\rfoot{}
\renewcommand{\headrulewidth}{0.4pt}
\renewcommand{\footrulewidth}{0.4pt}

\newtcolorbox{mybox}[1]{enhanced, colback=GhostWhite, colframe=LightGray, coltitle=black,fonttitle=\bfseries\Large, bottomrule=1ex, breakable=true,title=#1}
\begin{document}
\noindent
零、D\\
一、\\
1-5 \quad AACAD\\
6-10\quad BACBD\\
二、\\
11.$\ln x$\\
12.$\frac{4\ln 2}{4-\sqrt{2}\pi}$\\
13.$\frac{\pi}{2e}$\\
14.$\frac{1}{2}e^{u}-\frac{1}{2}e^{-u} -u$\\
15.$x^2+y^2-13z^2-4x-6y-18z+3=0$\\
16.$2\cdot 3^n$\\
17.$130$\\
18.$9^{100}\begin{pmatrix}
    1&2&-2\\
    2&4&-4\\
    -2&-4&4\\
\end{pmatrix}$\\
19.$\frac{1}{3}$\\
20.$2\ln 2$\\
三、\\
21.\\
\textbf{解:}\\
(1)\\
\begin{align*}
	\lim \limits_{n\to\infty} (1+\sin(\pi\sqrt{4n^2+2}))^{n}&=\lim \limits_{n\to\infty} (1+\sin(\pi\sqrt{4n^2+2}-2n\pi))^{n}\\
	&=\lim \limits_{n\to\infty} (1+\sin(\frac{2\pi}{\sqrt{4n^2+2}+2n}))^{n}\\
	&=exp(\lim\limits_{n\to\infty} n\sin(\frac{2\pi}{\sqrt{4n^2+2}+2n}))\\
	&=exp(\lim\limits_{n\to\infty} \frac{2n\pi}{\sqrt{4n^2+2}+2n})\\
	&=e^{\frac{2\pi}{4}}\\
	&=e^{\frac{\pi}{2}}
\end{align*}
(2)\\
\begin{align*}
    &\quad \lim\limits_{x\to 0}\left[\frac{\arctan (1-e^x)}{\sqrt{1+4x^2}}+(\frac{\arctan x}{\tan x})^\frac{3}{\sin^2 x}\right]\\
    &=\frac{0}{1}+ \lim\limits_{x\to 0}(1+\frac{\arctan x-\tan x}{\tan x})^{\frac{3}{\sin^2 x}}\\
    &=exp(\lim\limits_{x\to 0} \frac{3(\arctan x -\tan x)}{\tan x \sin^2 x})\\
    &=exp(\lim\limits_{x\to 0} \frac{3(-\frac{2}{3}x^3)}{x^3})\\
    &=\frac{1}{e^2}
\end{align*}
(3)\\
\begin{align*}
	&\quad\lim\limits_{x \to 0} \frac{\tan(\tan x)-\sin(\sin x)}{x-\sin x}\\&= \lim \limits_{x \to 0} \frac{\tan(\tan x) -\tan x +\tan x -\sin x + \sin x -\sin(\sin x)}{x-\sin x}\\
	&=\lim \limits_{x\to 0}\frac{\frac{1}{3}\tan^3 x +\frac{1}{2}x^3 +\frac{1}{6}\sin^3 x}{\frac{1}{6}x^3}\\
	&=6
\end{align*}
(4)\\
\begin{align*}
	\lim\limits_{x\to 0^{+}} \frac{x^x -(\sin x)^x}{x^2\ln(1+x)}&=
	\lim\limits_{x\to 0^{+}} x^x \lim\limits_{x\to 0^{+}} \frac{1-(\frac{\sin x}{x})^x}{x^3}\\
	&=\lim\limits_{x\to 0^{+}} \frac{1-e^{x\ln(1+\frac{\sin x - x}{x})}}{x^3}\\
	&=\lim\limits_{x\to 0^{+}} \frac{x-\sin x}{x^3}\\
	&=\frac{1}{6}
\end{align*}
22.\\
\textbf{解:}\\
(1)\\
$(\sqrt{x^2+y^2}-b)^2+z^2=a^2$\\
(2)\\
$z=\pm \sqrt{a^2-(r-b)^2}$
\begin{align*}
    &\quad \iiint\limits_{\Omega}(x+y)^2\mathrm{d}v\\
    &=\int_{0}^{2\pi} \mathrm{d} \theta \int_{b-a}^{b+a} r\mathrm{d}r\int_{-\sqrt{a^2-(r-b)^2}}^{\sqrt{a^2-(r-b)^2}} (r\cos\theta+r\sin \theta)^2\mathrm{d}z\\
    &=\int_{0}^{2\pi}(1+2\sin x\cos x)\mathrm{d}\theta \int_{b-a}^{b+a} 2r^3\sqrt{a^2-(r-b)^2}\mathrm{d} r\\
    &=2\pi \int_{-a}^{a}2(u+b)^3\sqrt{a^2-u^2}\mathrm{d}u\\
    &=4\pi\int_{-a}^{a}(u^3+3u^2b+3ub^2+b^3)\sqrt{a^2-u^2}\mathrm{d}u\\
    &=8\pi\int_{0}^{a}(3u^2b+b^3)\sqrt{a^2-u^2}\mathrm{d}u\\
    &=8\pi(\frac{b^3\pi a^2}{4}+\int_{0}^{\frac{\pi}{2}}3a^2 b\sin^2 t\ a\cos t\ a\cos t\ \mathrm{d} t)\\
    &=2a^2 b^3\pi^2+8\pi \int_{0}^{\frac{\pi}{2}} 3a^4b \sin^2 t\ \cos^2 t \mathrm{d}t\\
    &=2a^2 b^3\pi^2+8\pi \int_{0}^{\frac{\pi}{2}} 3a^4b (\sin^2 t-\sin^4 t)\mathrm{d}t\\
    &=2a^2 b^3\pi^2+8\pi\ 3a^4b(\frac{1}{2}\frac{\pi}{2}-\frac{3}{4}\frac{1}{2}\frac{\pi}{2})\\
    &=2a^2 b^3\pi^2+\frac{3a^4 b \pi^2}{2}
\end{align*}
23.\\
\textbf{解:}
(1)\\
不妨设$x_1<x_2$\\
$min(f'(x_1),f'(x_2))\le \sqrt{f'(x_1)f'(x_2)} \le max(f'(x_1),f'(x_2))$\\
由介值定理$\exists c\in\left[x_1,x_2\right]\subset (a,b)$,使得$f'(c)=\sqrt{f'(x_1)f'(x_2)}$\\
(2)\\
由拉格朗日中值定理
\begin{align*}
    f\left[f(a)\right]-f\left[f(b)\right]&=f'(x_1) (f(a)-f(b)) \left(x_1\text{介于}f(a),f(b)\text{之间,即}x_1\in(a,b)\right)\\
    f\left[f(a)\right]-f\left[f(b)\right]&=f'(x_1)f'(x_2)(a-b)\left(x_1,x_2 \in (a,b)\right)\\
    f\left[f(a)\right]-f\left[f(b)\right]&=\left[f'(\xi)\right]^2(a-b)\left(\xi \in (a,b)\right)
\end{align*}
24.\\
\textbf{解:}
$$\begin{pmatrix}
    x_n\\
    y_n\\
\end{pmatrix}=\begin{pmatrix}
    1&2\\
    4&3\\
\end{pmatrix}\begin{pmatrix}
    x_{n-1}\\
    y_{n-1}\\
\end{pmatrix}=\begin{pmatrix}
    1&2\\
    4&3\\
\end{pmatrix}^n\begin{pmatrix}
    x_{0}\\
    y_{0}\\
\end{pmatrix}=\begin{pmatrix}
    1&2\\
    4&3\\
\end{pmatrix}^n\begin{pmatrix}
    2\\
    1\\
\end{pmatrix}$$
$$A=\begin{pmatrix}
    1&2\\
    4&3\\
\end{pmatrix}$$
设$\lambda_1,\lambda_2$为$A$的特征值\\
$$\left| \lambda E -A\right| =\begin{pmatrix}
    \lambda-1&-2\\
    -4&\lambda -3\\
\end{pmatrix}=\lambda^2-4\lambda -5=(\lambda-5)(\lambda +1)=0$$
$$\Rightarrow \lambda_1=5,\lambda_2=-1$$
设$\xi_1,\xi_2$分别为$\lambda_1,\lambda_2$对应的特征向量\\
$(A-5E)\xi_1=0\Rightarrow \xi_1=\begin{pmatrix}
    1\\
    2\\
\end{pmatrix}$\\
$(A+E)\xi_2=0 \Rightarrow \xi_2=\begin{pmatrix}
    1\\
    -1\\
\end{pmatrix}$\\
$$P=\begin{pmatrix}
    1&1\\
    2&-1\\
\end{pmatrix}$$
$$P^{-1}=\frac{1}{-1-2}\begin{pmatrix}
    -1&-1\\
    -2&1
\end{pmatrix}=\frac{1}{3}\begin{pmatrix}
    1&1\\
    2&-1
\end{pmatrix}$$
$$P^{-1}AP=diag(5,-1)$$
\begin{align*}
    A^{2077}&=(P\ diag(5,-1)\ P^{-1})^{2077}\\
    &=\frac{1}{3} \begin{pmatrix}
        1&1\\
        2&-1
    \end{pmatrix} \begin{pmatrix}
        5^{2077}&0\\
        0&-1
    \end{pmatrix}\begin{pmatrix}
        1&1\\
        2&-1
    \end{pmatrix}\\
    &=\frac{1}{3} \begin{pmatrix}
        5^{2077}&-1\\
        2\cdot 5^{2077}&1\\
    \end{pmatrix} \begin{pmatrix}
        1&1\\
        2&-1
    \end{pmatrix}\\
    &=\frac{1}{3}\begin{pmatrix}
        5^{2077}-2&5^{2077}+1\\
        2\cdot 5^{2077}+2&2\cdot 5^{2077}-1\\
    \end{pmatrix}
\end{align*}
$$\begin{pmatrix}
    x_{2077}\\
    y_{2077}\\
\end{pmatrix}=A^{2077}\begin{pmatrix}
    2\\
    1\\
\end{pmatrix}=\frac{1}{2}\begin{pmatrix}
    2\cdot 5^{2077} -4+5^{2077}+1\\
    4\cdot 5^{2077}+4+2\cdot 5^{2077}-1\\
\end{pmatrix}=\begin{pmatrix}
    5^{2077} -1\\
    2\cdot 5^{2077}+1\\
\end{pmatrix}$$
$$\Rightarrow x_{2077}=5^{2077} -1$$
25.\\
\textbf{解:}\\
(1)\\
\begin{align*}
    (A-E)\alpha_3-\alpha_2&=0\\
    A\alpha_3=\alpha_2+\alpha_3
\end{align*}
$\alpha_1,\alpha_2$对应$A$的不同特征值的特征向量,所以线性无关\\
设$k_1 \alpha_1+k_2\alpha_2+k_3\alpha_3=0\cdots (1$)\\
$A(k_1 \alpha_1+k_2\alpha_2+k_3\alpha_3)=-k_1\alpha_1+k_2\alpha_2+k_3\alpha_2+k_3\alpha_3=0\cdots (2)$\\
(1)-(2)\\
$2k_1\alpha_1 -k_3\alpha_2=0$\\
因为$\alpha_1,\alpha_2$线性无关,所以$k_1=k_3=0$\\
$\Rightarrow k_2=0$\\
所以$\alpha_1,\alpha_2,\alpha_3$线性无关\\
所以$P$可逆\\
(2)\\
$$Q=\begin{pmatrix}
    -1&0&0\\
    0&1&1\\
    0&0&1\\
\end{pmatrix}$$
$$AP=PQ\Rightarrow P^{-1}AP=Q \Rightarrow A \sim Q$$
$$Q\text{特征值}-1,1,1$$
$$\left|A\right|=-1$$
$$A^{*}\text{特征值}1,-1,-1$$
$$P^{-1}A^{*}P=\begin{pmatrix}
    1&0&0\\
    0&-1&1\\
    0&0&-1\\
\end{pmatrix}$$
26.\\
\textbf{解:}
(1)\\
$$f_{\left. Y\right| X}(\left. y\right| x)=\begin{cases}
    \frac{1}{2x},-x< y < x,\\
    0,\text{其他}\\
\end{cases}$$
$$f(x,y)=f_{X}(x) f_{\left. Y\right| X}(\left. y\right| x)=\begin{cases}
    1,0<x<1,-x< y < x,\\
    0,\text{其他}
\end{cases}$$
$$f_{Y}(y)=\begin{cases}
    \int_{y}^{1}\mathrm{d}x,0< y < 1\\
    \int_{-y}^{1}\mathrm{d}x, -1< y <0\\
\end{cases}=\begin{cases}
    1-y,0<y<1\\
    1+y,-1<y<0\\
\end{cases}=1-\left|y\right| (-1<y<1)$$
$$EY=\int_{-1}^{1}y(1-\left|y\right|)\mathrm{d}y=0$$
\begin{align*}
    P\left\{\left. \frac{1}{2}<X<\frac{3}{2} \right|Y=EY\right\}
    &=P\left\{\left. \frac{1}{2}<X<\frac{3}{2} \right|Y=0\right\}\\
    &=\int_{\frac{1}{2}}^{\frac{3}{2}}\frac{f(x,y)}{f_{Y}(y)}\mathrm{d}x\\
    &=\int_{\frac{1}{2}}^{1}\mathrm{d}x\\
    &=\frac{1}{2}
\end{align*}
(2)\\
$$f_{X}(x)f_{Y}(y)\neq f(x,y),\Rightarrow \text{X,Y 不独立}$$
$$E(XY)=\int_{0}^{1}\mathrm{d} x \int_{-x}^{x} xy\mathrm{d}{y} =0$$
% $$E(X)=\int_{0}^{1}2x^2\mathrm{d}x=\frac{2}{3}$$
$$Cov(X,Y)=E(XY)-E(X)E(Y)=0\Rightarrow \text{不相关}$$
(3)\\
$$\begin{cases}
    z=x-y\\
    y=-x\\
\end{cases} \Rightarrow x=\frac{z}{2}$$
$$\text{当}0<z<2\text{时}$$
\begin{align*}
    F_{Z}(z)&=P(Z\le z)\\
    &=P(X-Y\le z)\\
    &=\int_{\frac{z}{2}}^{1}\mathrm{d}x\int_{-x}^{x-z}\mathrm{d}y\\
    &=\int_{\frac{z}{2}}^{1} (2x-z)\mathrm{d}x\\
    &=\left. (x^2-zx)\right|_{\frac{z}{2}}^{1}\\
    &=1-\frac{z^2}{4}-z+\frac{z^2}{2}\\
    &=\frac{z^2}{4}-z+1
\end{align*}
$$\frac{\mathrm{d} F_{Z}(z)}{\mathrm{d}z}=\frac{z}{2}-1$$
$$f_Z(z)=\begin{cases}
    \frac{z}{2}-1,0<z<2\\
    0,\text{其他}
\end{cases}$$
\newpage
粉丝牌\\
\includegraphics{fan_club.PNG} 
\end{document}